% !TeX document-id = {631e44a8-cd91-47da-bf82-1571f8787194}
% !TeX TXS-program:compile = txs:///pdflatex/[--shell-escape]

\documentclass{class/thesis}

\usepackage{amssymb}
\usepackage{pifont}
\usepackage{makecell}
\usepackage[strict=true,autostyle=true]{csquotes}
\usepackage{multicol}
\usepackage{longtable}
\usepackage{rotating}[clockwise]

\usepackage{tikz}
\usetikzlibrary{shapes.geometric, shapes.arrows, matrix, fit, patterns}

\usepackage{pgfplots}
\usepackage{pgfplotstable}
\usepgfplotslibrary{statistics, groupplots}
\pgfplotsset{compat=1.10}

% from: https://tex.stackexchange.com/questions/117435/read-boxplot-prepared-values-from-a-table
\makeatletter
\pgfplotsset{
	boxplot prepared from table/.code={
		\def\tikz@plot@handler{\pgfplotsplothandlerboxplotprepared}%
		\pgfplotsset{
			/pgfplots/boxplot prepared from table/.cd,
			#1,
		}
	},
	/pgfplots/boxplot prepared from table/.cd,
	table/.code={\pgfplotstablecopy{#1}\to\boxplot@datatable},
	row/.initial=0,
	make style readable from table/.style={
		#1/.code={
			\pgfplotstablegetelem{\pgfkeysvalueof{/pgfplots/boxplot prepared from table/row}}{##1}\of\boxplot@datatable
			\pgfplotsset{boxplot/#1/.expand once={\pgfplotsretval}}
		}
	},
	make style readable from table=lower whisker,
	make style readable from table=upper whisker,
	make style readable from table=lower quartile,
	make style readable from table=upper quartile,
	make style readable from table=median,
	make style readable from table=lower notch,
	make style readable from table=upper notch
}
\makeatother

% We will externalize the figures
%\usepgfplotslibrary{external}
%\tikzexternalize

\newcommand{\cmark}{\ding{51}} %checkmark
\newcommand{\xmark}{\ding{55}} %x-cross
\renewcommand\theadfont{\bfseries\sffamily}

%%%%%%%%%%%%%%%%%%%%%%%%%%%%%%%%%%%%%%%%%%%%%%%%
% Titel der Arbeit                             %
%%%%%%%%%%%%%%%%%%%%%%%%%%%%%%%%%%%%%%%%%%%%%%%%
\thesistitle{Accelerating S3 by implementing caching via composable proxies}

%%%%%%%%%%%%%%%%%%%%%%%%%%%%%%%%%%%%%%%%%%%%%%%%
% Schlüsselwörter                              %
%%%%%%%%%%%%%%%%%%%%%%%%%%%%%%%%%%%%%%%%%%%%%%%%
\thesiskeywords{S3, Simple Storage System, Proxy}

%%%%%%%%%%%%%%%%%%%%%%%%%%%%%%%%%%%%%%%%%%%%%%%%
% Sprache der Arbeit                           %
%                                              %
%   ngerman  - Deutsch (neue Rechtschreibung)  %
%   english  - Englisch                        %
%%%%%%%%%%%%%%%%%%%%%%%%%%%%%%%%%%%%%%%%%%%%%%%%
\thesislanguage{english}

%%%%%%%%%%%%%%%%%%%%%%%%%%%%%%%%%%%%%%%%%%%%%%%%
% Art der Arbeit                               %
%                                              %
%   bachelor  - Bachelorarbeit                 %
%   master    - Masterarbeit                   %
%   project   - Projektarbeit                  %
%%%%%%%%%%%%%%%%%%%%%%%%%%%%%%%%%%%%%%%%%%%%%%%%
\thesistype{bachelor}

%%%%%%%%%%%%%%%%%%%%%%%%%%%%%%%%%%%%%%%%%%%%%%%%
% Ausgabe von Verzeichnissen                   %
%                                              %
%   true   - Aktiviert                         %
%   false  - Deaktiviert                       %
%%%%%%%%%%%%%%%%%%%%%%%%%%%%%%%%%%%%%%%%%%%%%%%%
\listofalgorithmsenabled{false}
\listoffiguresenabled{true}
\listoftablesenabled{true}

%%%%%%%%%%%%%%%%%%%%%%%%%%%%%%%%%%%%%%%%%%%%%%%%
% Vollständiger Name                           %
%%%%%%%%%%%%%%%%%%%%%%%%%%%%%%%%%%%%%%%%%%%%%%%%
\thesisauthor{Matthias A. Metzen}

%%%%%%%%%%%%%%%%%%%%%%%%%%%%%%%%%%%%%%%%%%%%%%%%
% Geburtsort                                   %
%%%%%%%%%%%%%%%%%%%%%%%%%%%%%%%%%%%%%%%%%%%%%%%%
\authorbirthplace{Mettmann}

%%%%%%%%%%%%%%%%%%%%%%%%%%%%%%%%%%%%%%%%%%%%%%%%
% Datum der Abgabe                             %
%%%%%%%%%%%%%%%%%%%%%%%%%%%%%%%%%%%%%%%%%%%%%%%%
\submissiondate{24. July 2023}

%%%%%%%%%%%%%%%%%%%%%%%%%%%%%%%%%%%%%%%%%%%%%%%%
% Erstgutachter                                %
%%%%%%%%%%%%%%%%%%%%%%%%%%%%%%%%%%%%%%%%%%%%%%%%
\firstreviewer{Prof. Dr. Michael Schöttner}

%%%%%%%%%%%%%%%%%%%%%%%%%%%%%%%%%%%%%%%%%%%%%%%%
% Zweitgutachter                               %
%%%%%%%%%%%%%%%%%%%%%%%%%%%%%%%%%%%%%%%%%%%%%%%%
\secondreviewer{Prof. Dr. Martin Mauve}

%%%%%%%%%%%%%%%%%%%%%%%%%%%%%%%%%%%%%%%%%%%%%%%%
% Betreuer                                     %
%%%%%%%%%%%%%%%%%%%%%%%%%%%%%%%%%%%%%%%%%%%%%%%%
\supervisor{Prof. Dr. Michael Schöttner}
\bibliography{references.bib}

\begin{document}
  \begin{thesis}
    \chapter{Introduction}

\section{Motivating Scenario}

\section{Problem Description and Scope}
\begin{enumerate}
\item Support all S3 requests
\item Focus on MinIO
\item Easy configurability
\end{enumerate}
    %\chapter{Structure}
\label{Structure}

Eine sinnvolle und korrekte Unterteilung der Arbeit ist nicht nur wichtig für den Leser, sondern hilft auch dem Verfasser bei der Anfertigung.\\
Allgemeiner grundlegender Aufbau mit den wichtigsten inhaltlichen Aspekten:

\begin{enumerate}
	\item Abstract
	\item Introduction
	\begin{enumerate}
		\item Problem Description
		\item Motivation
		\item Goals
	\end{enumerate}
	\item Foundation
	\begin{enumerate}
		\item Related Work
		\item Simple Storage Service
		\item Composable Proxy
	\end{enumerate}
	\item Architecture
	\item Implementation
	\begin{enumerate}
		\item Decisions
		\begin{enumerate}
			\item Modularity
			\item Configurability
			\item HTTP Cache Headers
			\item ACL
		\end{enumerate}
		\item Components
		\begin{enumerate}
			\item Server
			\item Middleware
			\item Client
			\item Webhook
		\end{enumerate}
		\item Challenges: s3s
	\end{enumerate}
	\item Evaluation
	\begin{enumerate}
		\item Benchmark Results
		\begin{enumerate}
			\item Same Server
			\item Clientside Proxy, Remote Server
			\item Proxy Tunnel, Remote Server
		\end{enumerate}
		\item Bandwidth
	\end{enumerate}
	\item Summary
	\item Future Work
\end{enumerate}
    \chapter{Fundamentals}

\section{Simple Storage Service}
\label{S3_desc}
Simple Storage Service (S3) is a cloud storage service originally developed by Amazon Web Services (AWS) that offers reliable and scalable cloud storage for data through a HTTP-based API \cite{AWS_S3_Release} \cite{S3_API}. Since its initial release in 2006, many alternative implementations and competing S3-compatible services have been developed. Although the term S3 originally only referred to the service offered by AWS, it has now often used as a general term for any compatible solution.
In S3, individual data entries are called \textit{objects}, which consist of the \textit{value}, which can consist of any sequence of bytes ranging from 0 to 5 TB in size\cite{AWS_S3_FAQ}, and \textit{metadata}, a set of key-value pairs that stores additional information about the object. Objects are grouped in a container structure called \textit{buckets}, where each object belongs to exactly one bucket, and every bucket can contain any number of objects. Within a bucket, each object is uniquely identified through a \textit{key} and a \textit{version ID}.
The data stored in S3 can be accessed and managed through an HTTP-based REST API, which currently defines 97\cite{S3_API} operations through which data can be accessed and manipulated.
Access to the REST API is configured through policies that can be defined for each bucket which enable granular control over which operation can be executed by which user. The way user accounts are managed differs between implementations, but generally, it can be said that users will be associated with \textit{credentials} consisting of an \textit{access key id} and a \textit{secret key}. To authorize requests to the API, users need to provide a signature for each request. This signature is calculated using multiple points of data like the contents of the request, a timestamp at which the request was made, as well as the user's credentials. The signature can be sent either through the request's \textbf{Authorization} header or its URL and is validated by the server.

With the ever-increasing amount of data that companies manage and the inherent need for storage capacity, S3 offers an easily manageable abstraction where developers do not have to think about the underlying filesystems and storage mediums. The decoupling of storage and access also makes it possible to expand storage capacity by spreading objects over multiple devices and even adding more capacity when necessary, allowing for near-infinite scalability. With managed solutions like the one provided by AWS, pricing is commonly calculated only by the amount of storage in use, which allows users to build solutions that can effortlessly scale to large amounts of data without wasting money on unused capacity caused by overprovisioning.

\section{MinIO}
MinIO is an implementation of a S3-compatible object store. It supports most of the S3 core features. Its greatest advantage is that it offers various means to create self-hosted instances, making it not only versatile but also favored by companies in regard to data protection.

\section{Caching}

Caching is an important technique in computing to enhance the performance of computer systems. By storing frequently accessed data in a fashion that is faster to access, both the latency and throughput of applications can be greatly improved.
This section gives a brief overview of some of the terms used to describe caching strategies and outlines the cache policies used within this work.

\subsection{Admission Policy}
The admission policy defines the logic employed by a caching system to determine whether a resource should be stored in the cache or not. It can be influenced by various factors, such as the cache capacity, the size of a resource or its relevance.

\subsection{Eviction Policy}
The eviction policy describes the mechanism by which a caching system selects items to be removed from the cache when new items are inserted into a cache that has already reached its maximum capacity. Popular eviction policies include the Least Recently Used (LRU), and Least Frequently Used (LFU) strategies.

\subsection{TinyLFU}
\label{tiny_lfu}
TinyLFU is \textquote[{\cite{einziger2015tinylfu}}]{an approximate frequency based cache admission policy \textelp{that} can augment caches of arbitrary eviction policy and significantly improve their performance. \textelp{T}he memory consumption of TinyLFU \textins{was optimized} using adaptation of known approximate counting techniques with novel techniques tailored specifically in order to achieve low memory footprint for caches.}
In TinyLFU \textquote[{\cite{einziger2015tinylfu}}]{an accessed item is only inserted into the cache if an admission policy decides that the cache hit ratio is likely to benefit from replacing it with the cache victim \textelp{} chosen by the cache’s replacement policy}.

\section{HTTP Caching}
\label{http_cache_policy}

HTTP (Hypertext Transfer Protocol) caching is an important mechanism that allows web clients and intermediaries to store and reuse previously requested responses. Caching plays a vital role in improving the responsiveness of web applications and reducing bandwidth usage. To control the caching behavior of web resources, a set of HTTP headers were first defined as part of the HTTP/1.1 protocol\cite{rfc2616}. These headers provide instructions to clients and intermediate caching servers on how to handle and store cached content.

This section provides an overview of the most commonly used HTTP headers related to caching and their role in cache applications based on their definitions in RFC9111 \cite{rfc9111}.

\subsection{Cache-Control} The Cache-Control header is the most fundamental directive that defines caching behavior. It specifies whether a resource can be cached, for how long it can be cached and whether a resource is allowed to be stored in a shared cache or only in a private cache. This header provides various options such as "no-cache" (attempt to validate the resource on every request), "no-store" (a cache must not store any part of the request or the response), "max-age" (set the maximum amount of time a resource may be cached), and "public" (allow caching in shared caches).

\subsection{Expires} The Expires header indicates a timestamp after which a response should be considered stale. If a Cache-Control header with the "max-age" directive is present, the Expires header will be ignored.

\subsection{ETag} The ETag header provides a unique identifier that a server can assign to a specific version of a resource. It enables the server or cache to determine if a cached resource is still valid or if it conditionally needs to be refreshed. By augmenting subsequent requests to a resource with the If-None-Match header containing the previous ETag, the server can avoid unnecessary data transfer by responding with a "304 Not Modified" status.

\subsection{Last-Modified} The Last-Modified header provides a timestamp of when a resource was last modified on the server. By augmenting subsequent requests with the Last-Modified-Since header containing the previous Last-Modified value, the server can determine if a cached resource is still valid and conditionally responds with a "304 Not Modified" status.

\subsection{Vary} Servers use the Vary header to let intermediaries and caches know that the response might change depending on the values of specific request headers. It defines a set of headers that must match between subsequent requests and the original request used to retrieve a specific resource. If any of the headers do not match, the cached response may not be used to serve the request.
    \chapter{Architecture}
\label{architecture}

This section gives an overview of the general architecture and request handling of the software. Most of the internal design is described in terms of abstract \textit{components}, which form the building blocks of the software. There can be multiple implementations for components of any type that provide different functionality from each other.
To provide a high degree of flexibility and extensibility, all components are only very loosely coupled, with very limited interaction points between individual components. This is achieved by defining a unified internal representation for requests and responses. All components involved in request handling have their API defined in terms of these two types, thus allowing components to easily be combined or new components to be added without breaking compatibility. For easier handling, both types can also be extended with additional data to be passed between components.
\\\\
Requests are generated by a \code{Server} component, which serves as an entrypoint to the proxy. The protocol and format through which external requests can be made to a \code{Server} are implementation specific to a given \code{Server}. When an external request is made to a \code{Server}, the server creates the internal representation from it. It then calls a \textit{Handler} function that it has been passed. The handler function returns a future that asynchronously resolves the request. Upon completing the future, the server then transforms the resolved response into a response accepted by its underlying protocol and resolves the incoming request.
\\\\
Through the \code{Handler} function, the request is passed to a number of \code{Middleware} components. These \code{Middleware}s are responsible for operating on the request and providing additional capabilities.
Middlewares have access to both the request and response of every request and can intercept, replace or modify them.
The cache, one of the focuses of this work, has been implemented as one such middleware. Upon receiving the request, it checks for the cachability and attempts to retrieve an appropriate response from the cache. The cachability and validity of a cached response are validated using standardized HTTP cache policies. In case a matching response was found, the middleware short-circuits and no further middleware will be executed. 
\\\\
If the request has successfully passed through the middleware stack, it is passed to a \code{Client} component, which is responsible for resolving the request into a response. Like the \code{Server} component, the method used to resolve the request is dependent on the specific implementation and does not need to match the \code{Server} component that was used to accept the request.
\\\\
To enable additional interaction with the proxy, in addition to the described request-response pattern a \code{Webhook} component can be configured to receive external events. These events are broadcast to all active components, allowing them to further enhance their capabilities.
\\\\

\begin{figure}
	\label{diagram_architecture}
	\centering
	\begin{tikzpicture}
		\matrix (m) [matrix of nodes,
		column sep=5mm,
		row sep=1cm,
		nodes={draw, % General options for all nodes
			line width=1pt,
			anchor=center, 
			text centered,
			rounded corners,
			minimum width=1.5cm, minimum height=8mm
		}, 
		% Define styles for some special nodes
		broadcast/.style={cylinder, sharp corners, scale=0.5},
		other/.style={ellipse, scale=0.5},
		server/.style={isosceles triangle, anchor=center},
		middleware/.style={rectangle, rotate=90, scale=0.8},
		client/.style={dart},
		webhook/.style={trapezium, shape border rotate=90},
		empty/.style={draw=none, minimum width=0mm}
		]
		{
			|[empty]| {}
			& |[empty]| {}
			& |[empty]| {}
			& |[empty]| {}
			& |[empty]| {}
			& |[broadcast]| {Broadcast}
			& |[webhook]| {Webhook} 
			& |[empty]| {}
			\\
			|[empty]| {}
			& |[server]| {Server}
			& |[middleware]| {Middleware}
			& |[empty]| {...}
			& |[middleware]| {Middleware}
			& |[client]| {Client}
			& |[empty]| {}
			& |[empty]| {}
			\\
		};
		
		{
			\draw[<->, dashed, line width=1pt] (m-2-1.west) -- (m-2-2.west);
			\path[->, line width=1pt] (m-2-2) edge (m-2-3);
			\path[line width=1pt] (m-2-3) edge (m-2-4);
			\path[->, line width=1pt] (m-2-4) edge (m-2-5);
			\path[->, line width=1pt] (m-2-5) edge (m-2-6);
			\draw[<->, dashed, line width=1pt] (m-2-6.east) -- (m-2-8.east);
			
			\draw[<-, line width=1pt, rounded corners] (m-1-7.west) -- (m-1-6.east);
			\draw[<->, line width=1pt, rounded corners] (m-1-6.west) -| (m-2-2.north);
			\draw[<->, line width=1pt, rounded corners] (m-1-6.west) -| (m-2-3.east);
			\draw[<->, line width=1pt, rounded corners] (m-1-6.west) -| (m-2-5.east);
			\draw[<->, line width=1pt, rounded corners] (m-1-6.south) -- (m-2-6.north);
			
			\draw[<-, dotted, line width=1pt, rounded corners] (m-1-7.east) -- (m-1-8.east);
			\draw[->, dotted, line width=1pt, rounded corners] (m-1-6.west) -- (m-1-1.west);
			
			\tikzset{blue dotted/.style={draw=blue!50!white, line width=1pt,
					dash pattern=on 1pt off 4pt on 6pt off 4pt,
					inner sep=5mm, rectangle, rounded corners}};
				
			\node (pipeline box) [blue dotted, fit = (m-2-2) (m-1-7)] {};
			\node at (pipeline box.north) [below, inner sep=3mm] {\textbf{Pipeline}};
			
			\node (stack box) [blue dotted, fit = (m-2-3) (m-2-5)] {};
			\node at (stack box.north) [below, inner sep=3mm] {\textbf{Stack}};
		
		};
		
		\node (downstream) [rectangle, draw, rotate=90, left=of m, minimum width=6cm, anchor=center] {Downstream};
		\node (downstream) [rectangle, draw, rotate=90, right=of m, minimum width=6cm, anchor=center] {Upstream};
	\end{tikzpicture}
	\caption{Diagram of the architecture}
\end{figure}
    %\chapter{Selecting a S3 library}

This chapter focuses on the necessity of utilizing a library for the implementation of the S3-specific components of the software described and presents thought and evaluations for a selection of libraries available. It then briefly describes the library chosen and outlines some challenges posed by the library as well as some modifications that were applied to it as part of this work. The main criteria of the evaluation were the ease of integration, the flexibility and, if necessary, the maintainability of modifications.

The use of a library significantly reduces the maintenance and increases the compatibility with different implementations of the S3 specification. The usage of a library should provide or significantly simplify the implementation of the following features:
\begin{enumerate}
	\item Parsing, validation and generation of request signatures (\ref{S3_desc})
	\item Assigning requests to the corresponding API operations
	\item Parsing the contents of HTTP requests \& responses
	\item \label{types_deser} Deserialization of requests \& responses to appropriate Rust data types and serialization thereof
\end{enumerate}

Although 1. and 2. could have been manually implemented as part of this work, for 3. and 4. it was unreasonable to do so, as per the stated goal of implementing all 97 S3 operations this would require implementing parsing and (de)serialization for 184 distinct and mostly non-trivial types.

\section{aws-sdk-s3}

aws-sdk-s3 is part of the AWS SDK for Rust, a comprehensive set of libraries provided by AWS that provide integrations for many AWS services and is code generated using a code generator that consumes JSON files which describe the AWS services according to the Smithy Model specification\cite{AWS_SDK_GITHUB}. It is widely used and commonly accepted as the default library to interact with S3 services\cite{} and actively developed and maintained by AWS, making it a good choice for maintainability. For each API operation it defines a typed representation for both the request and response associated with that operation. Additionally, it provides a complete client implementation for S3 APIs that implements request signatures as well as serialization and deserialization of the provided data types through its integration with other parts of the AWS SDK. 

Although containing all of the features outlined at the beginning of this chapter, further evaluation showed that the design of aws-sdk-s3 posed significant challenges to its integration for this work. Most significantly, the implementation for the serialization and deserialization of its data types is encapsulated inside the client implementation and not exposed to users of the library.


TODO: 
\begin{itemize}
	\item Data types can only be initiated through a builder pattern\ref{builder_pattern}
	\item Private fields, no setters
\end{itemize}

\section{s3s}

\subsection{Modifications}
\subsubsection{Private functions}
\label{s3s_mod_pub}
\subsubsection{Operation type handling}
\label{s3s_mod_op}
\subsubsection{Copying metadata}
\label{s3s_mod_meta}
    \chapter{Implementation}

\section{Decisions}

\subsection{Modularity}

\subsection{Configurability}

\subsection{Security}
\label{security}

Since requests to S3 are usually authenticated through a signature unique to each request there needed to be considerations on if and how these requests can be safely cached without exposing data to unauthorized access.

There were plans to implement bucket policy validations inside the proxy. This would enable failing early in the case that a request was sent with insufficient permission, reducing bandwidth and server load. However, there are some major issues with this idea.
Most significantly, a full permissions check would require access to the credentials of all users. Exposing all credentials to the proxy would pose a significant risk and increase the attack surface. As there is no API access to the credentials defined in the S3 specification, the implementation would also not be universally applicable to all S3-compatible services.
A solution would be to only make the proxy accessible through a set of known credentials against which it could validate the bucket policies. After careful evaluation, this solutions was ultimately judged to be out of scope for this work, as in this case requesters can be assumed to have knowledge of the proxy and therefore can be expected to have the correct credentials, in which case there would be limited room for acceleration.

The solution that was eventually adopted is that the proxy can optionally be configured to validate incoming requests against a provided set of credentials (A), but will replace the signature with one calculated from its own credentials (B) before forwarding the request.
Although acceptable within the scope of this work, this approach does come with some issues and poses some limitations that should be considered. 
\begin{enumerate}
	\item If A is configured and B has permissions that A does not have, A can bypass this lack of permissions. Conversely, if A has permissions that B does not have, downstream requests may fail unexpectedly. This can be mitigated by using the same set of credentials for A and B.
	\item If A is configured, a requests signature must match A even if the bucket policy would allow for public access.
	\item If no credentials are configured for A, all requests to the proxy will be executed with B without any validation of signatures on the incoming request. Exposing the proxy in this state poses significant risk of unintended unauthorized access.
\end{enumerate}

\section{Components}

This section focuses on the specific implementations for the components described in \ref{architecture} that were created for the stated goal of creating a proxy for the S3 protocol with additional cache functionality.

\subsection{S3 Server}

The Server is the component responsible for accepting requests made to the proxy. It is initialized by providing a ServerBuilder with a Handler to which it will forward all accepted requests.

The S3Server is built upon a traditional HTTP server provided by hyper

\subsection{Middleware}

\subsubsection{Caching}
\begin{enumerate}
	\item Cache policy: TinyLFU
	\item Size aware eviction
	\item Headers not counted towards size
	\item global Time-To-Live and Time-To-Idle + configurable per operation
\end{enumerate}

\subsection{S3 Client}

\subsection{Pipeline}

The Pipeline is a management structure composed of a ServerBuilder, any number of Middlewares and a Client. It is responsible for assembling the different components passed to it into a cohesive unit. When initialized with all required components it first constructs the RequestProcessor from the client and middlewares together with a broadcast channel that is registered with all components. The broadcast channel is used to pass events other than requests, like the ones produced by the Webhook component. It then converts the RequestProcessor into a Handler, which is then used to finalize the ServerBuilder to start the Server.

\subsection{Webhook}

\section{Challenges}

\subsection{smithy}
\subsection{s3s}
    \chapter{Evaluation}

In this chapter, the evaluation of the proposed caching proxy implementation for S3 storage systems is presented. The main objective of this evaluation is to assess the performance and effectiveness of the caching proxy in terms of throughput, latency, and savings in network bandwidth compared to the unproxied access to the underlying S3 instance.
This is done by evaluating and comparing multiple performance metrics measured in a variety of situations for both the imlpementation presented by this work to its' connected storage server. For the latter self-hosted instances of MinIO are used to provide a reproducable baseline.

\section{Benchmark Tools and Data Collection}

The purpose of this section is to introduce the benchmark tools which are utilized to evaluate this implementation of a caching proxy. These tools were chosen for their capabilities in accurately measuring and analyzing the performance of S3 storage systems and HTTP-based servers.

\subsection{S3 Benchmark Tool - Warp}
\label{warp}
Warp is a benchmark tool developed by the creators of MinIO and is specifically designed to evaluate the performance of S3 storage systems. The performance metrics of S3 operations, such as object retrieval and storage, are measured using a comprehensive selection of configurable tests. It provides detailed information on the throughput and latency of operations on large sets of existing data, or artificial data that it generates.
In this evaluation it was configured to query the list of existing objects from S3 which it then uses to create random access patterns by querying these objects. This tool was chosen over other S3 benchmarks specifically for its' functionality to randomly request objects multiple times so that the caching efficiency can be  observed. Many other S3 benchmarks were observed to only request objects once or in a fixed order which does not benefit from caching and does not reflect the reality of most applications for S3 data storage.

\subsection{Load Generation Tool: Oha}
\label{oha}
oha is a powerful HTTP load generation tool that allows the simulation of a high volume of requests to the caching proxy. By simulating multiple concurrent clients, oha generates high load on the server and provides precise data on the scalability and response time of the caching proxy. In order evaluate the performance of the cache under near ideal conditions, oha is used to test the performance of repeatedly retrieving only a single object, ensuring that the object should always remain in the cache.

\subsection{Bandwidth Monitoring Tool: iftop}

iftop\cite{iftop} is a command-line network monitoring software. It provides a real-time overview of network communications and can capture the bandwidth usage and total data transfer amount of connections. It was chosen amongst similar tools for its ability to limit the network interfaces it observes and the option to filter accounted connections based on a variety of factors. By filtering connections by interface and upstream port, it was possible to accurately measure the data transfer between systems.


\section{Evaluation Metrics}

The evaluation of the caching proxy implementation will be conducted based on three primary metrics:

\begin{description}[style=nextline]
\item[Throughput] Throughput refers to the number of requests processed by the server in a given amount of time. The efficiency of the caching proxy in handling multiple requests simultaneously is assessed through measuring the throughput under different workloads. This evaluation provides insights into the caching proxy's ability to handle varying levels of request concurrency. The amount of concurrent requests is calculated by multiplying the throughput with the latency.

\item[Latency] Latency measures the time taken by the server to respond to a request. It reflects the responsiveness of the system and is a crucial metric in evaluating the user experience. The analysis of latency for different object sizes allows for an understanding of how the caching mechanism affects the overall response time.

\item[Data transfer] Data transfter refers to the total amount of data transmitted between the caching proxy and its' upstream server. It is an essential metric for evaluating the efficiency of the caching mechanism. A reduction in the amount of network traffic between the proxy and the upstream server compared to unproxied access to the upstream server indicates a higher rate of cache hits, which results in fewer refetches, leading to lower latency and less load on the server. In configurations where network usage is billed this can also result in significant cost savings for many applications.

\end{description}

\section{Configurations evaluated}

All benchmarks used in this chapter were conducted on Hetzner CPX21 instances running Rocky Linux 9. The Hetzner CPX21 is equipped with 3 virtual CPUs and 4GB of RAM, providing a suitable environment for testing the capabilities of a lightweight S3 proxy.

For tests that involved interactions with \textit{remote} machines, another instance within the same datacenter was utilized. These instances were connected through a private network, ensuring a high-speed and low-latency connection for reliable and accurate performance measurements.

All instances of the proxy were configured to use a cache size of 500MB with a 300s expiration timer for both TTL and TTI. Additionaly, credential validation for the \code{S3Server} was disabled. This was necessary to enable evaluation using oha, which does not implement S3 signature creation.
\\\\
To evaluate the performance of s3p, the following configurations of proxies and S3 servers were tested:
\begin{description}[style=nextline] %[align=right,labelwidth=6em,leftmargin=6.5em]
	\item[${MinIO}_{local}$] The evaluation is performed on a MinIO instance running on the same machine as the benchmark. This serves as a baseline for local access.
	\item[${s3p}_{local}$] The evaluation is performed on an s3p instance running on the same machine as its' upstream MinIO instance. Comparing the results with ${MinIO}_{local}$ provides insight into the efficiency of the overall implementation and the additional latency added by the indirection.
	\item[${MinIO}_{remote}$] The benchmark is performed on a remote MinIO instance over the network. This serves as a baseline for requests sent over the network.
	\item[${s3p}_{remote}$] The evaluation is performed on a local s3p instance that is connected to a remote MinIO instance over the network. This is the most interesting configuration as it most accurately reflects the most common use-case for s3p.
	\item[${s3p}_{dual}$] The benchmark is performed on a local s3p instance that connects to a remote instance of s3p which then connects to a MinIO instance running on the same remote machine. For this configuration the two s3p instances were additionaly configured to communicate through HTTP/2, which was not possible for the previous configurations. This may bring additional performance and displays the \textit{external composition} capabilities of s3p.
\end{description}

The MinIO instances were preloaded with 2500 objects each for the sizes 1KiB, 10KiB, 100KiB, 500KiB, 1MiB that were randomly generated using warp.

\section{Results}

In this section, the benchmark results obtained from the evaluation of s3s are presented. The primary objective of the evaluation was to assess the cache efficiency under different workloads and the implementations' capabilities of accelerating S3 systems, using quantitative data to draw conclusions. For each metric this is done in two steps by first looking at the cases were s3p and MinIO were both running on the local machine, and then looking at the cases where communication happend over the network. The results provide insights into the system's strengths, weaknesses, and potential areas for improvement.

The results were obtained by conducting a separate run of warp and oha for every combination of the afore mentioned object sizes and configurations. To accurately measure the data transfer, separate recordings using the iftop tool were made for every individual run of warp and oha.

Results obtained through runs of warp are displayed with the plain name of the configuration as they are defined in the previous section, while results obtained through runs of oha are marked with a asterisk (eg. ${s3p}^{*}_{remote}$) to indicate that the result was obtained under near ideal conditions for caching by repeatedly querying only a single object.

\pgfplotscreateplotcyclelist{duo}{
	{blue,fill=blue!30!white,mark=none},%
	{red,fill=red!30!white,mark=none}
}

\pgfplotscreateplotcyclelist{trio}{
	{green,fill=green!30!white,mark=none},%
	{blue,fill=blue!30!white,mark=none},%
	{red,fill=red!30!white,mark=none}
}

\subsection{Throughput}

The goal of measuring throughput was to determine how much load s3p can take off of its' upstream server and to verify its capability of accelerating its' upstream server by providing additional concurrency and faster responses by storing them in fast, accessible memory through its' caching mechanism.

%local-minio
\pgfplotstableread{%
	colnames		0		1		2		3		4
	size			1 	   10 	  100	  500	 1000
	warp 		 2389	 2237	 1511	  678	  729
	oha			 3681	 3299	 1912	  816	  830
}\throughputlocmin
\pgfplotstabletranspose[colnames from=colnames]\Tthroughputlocmin{\throughputlocmin}

%local-s3p
\pgfplotstableread{%
	colnames		0		1		2		3		4
	size			1 	   10 	  100	  500	 1000
	warp		10060	 9699	 7826	  892	  795
	oha 		24589	24052	18518	 8945	 3771
}\throughputlocs
\pgfplotstabletranspose[colnames from=colnames]\Tthroughputlocs{\throughputlocs}

%local
\begin{figure}[h!]
	\centering
	\begin{tikzpicture}
		\begin{axis}[
			xlabel={Object size},
			ylabel={Objects/s},
			enlarge y limits={{0.1,upper}},
			xmin=0,
			ymin=0,
			xtick={1,10,100,500,1000},
			xmode = log,
			log basis x=10,
			scaled ticks=false,
			legend style={
				legend pos=outer north east,
				font=\small,
			},
			ymajorgrids=true,
			grid style=dashed,
			x tick label style = {font = \small, text width = 1.7cm, align = center, rotate = 70, anchor = north east},
			xticklabels={
				1KiB,
				10KiB,
				100KiB,
				500KiB,
				1MiB,
			},
			]
			
			\addplot+[
			cycle list name=duo,
			mark=square,
			]
			table [x=size,y=warp] {\Tthroughputlocs};
			
			\addplot+[
			cycle  list name=duo,
			mark=square,
			]
			table [x=size,y=warp] {\Tthroughputlocmin};
			
			\addplot+[
			cycle  list name=duo,
			mark=square,
			]
			table [x=size,y=oha] {\Tthroughputlocs};
			
			\addplot+[
			cycle  list name=duo,
			mark=square,
			]
			table [x=size,y=oha] {\Tthroughputlocmin};
			
			
			\legend{${s3p}_{local}$, ${MinIO}_{local}$, ${s3p}^{*}_{local}$, ${MinIO}^{*}_{local}$}
			
		\end{axis}
	\end{tikzpicture}
	\caption{Throughput of local s3p and MinIO instance}
	\label{fig:throughputlocal}
\end{figure}

The results for the \textit{local} scenario are provided in Figure \ref{fig:throughputlocal}. It shows that for small object sizes of 1-100KiB, s3p was able to provide a significant increase in throughput over MinIO, uplifting the number of objects served in that range by 4-5 times in the warp benchmark and by 6-9 times in the oha test. This indicates a very efficient cache usage with high hit rates in that range. The high hit rate within this range can primarily be explained by the fact that all 2500 objects contained in the dataset for the respective sizes can be held within the 500MB of allotted cache size simultaneously.

%remote-minio
\pgfplotstableread{%
	colnames		0		1		2		3		4
	size			1 	   10 	  100	  500	 1000
	warp 		 4239	 3833	 2477	  926	  493
	oha			 5907	 5352	 3123	  938	  400
}\throughputremmin
\pgfplotstabletranspose[colnames from=colnames]\Tthroughputremmin{\throughputremmin}

%remote-s3p
\pgfplotstableread{%
	colnames		0		1		2		3		4
	size			1 	   10 	  100	  500	 1000
	warp		10015	 9504	 7731	 1199	  575
	oha 		23808	23678	17038	 6629	 3445
}\throughputrems
\pgfplotstabletranspose[colnames from=colnames]\Tthroughputrems{\throughputrems}

%dual-s3p
\pgfplotstableread{%
	colnames		0		1		2		3		4
	size			1 	   10 	  100	  500	 1000
	warp		 9975	 9523	 7242	 1690	  638
	oha 		23701	22482	16924	 6240	 3101
}\throughputduos
\pgfplotstabletranspose[colnames from=colnames]\Tthroughputduos{\throughputduos}

%remote
\begin{figure}[h!]
	\centering
	\begin{tikzpicture}
		\begin{groupplot}[
			group style={
				group size= 2 by 1,
				horizontal sep=1em,
				yticklabels at=edge left,
			},
			xlabel={Object size},
			ylabel={Objects/sec},
			enlarge y limits={{0.1,upper}},
			xmin=0,
			ymin=0,
			xtick={1,10,100,500,1000},
			xmode = log,
			log basis x=10,
			ymajorgrids=true,
			grid style=dashed,
			scaled ticks=false,
			x tick label style = {font = \small, text width = 1.7cm, align = center, rotate = 70, anchor = north east},
			xticklabels={
				1KiB,
				10KiB,
				100KiB,
				500KiB,
				1MiB,
			},
			]
			
			\nextgroupplot[legend style={at={(1.05,1.05)}, font=\small}]
			%warp			
			\addplot+[
			cycle  list name=trio,
			mark=square,
			]
			table [x=size,y=warp] {\Tthroughputrems};
			
			\addplot+[
			cycle  list name=trio,
			mark=square,
			]
			table [x=size,y=warp] {\Tthroughputremmin};
			
			% oha		
			\addplot+[
			cycle  list name=trio,
			mark=square,
			]
			table [x=size,y=oha] {\Tthroughputrems};
			
			\addplot+[
			cycle  list name=trio,
			mark=square,
			]
			table [x=size,y=oha] {\Tthroughputremmin};
			
			\legend{${s3p}_{remote}$, ${MinIO}_{remote}$, ${s3p}^{*}_{remote}$, ${MinIO}^{*}_{remote}$}
			
			\nextgroupplot[ylabel=, legend style={at={(1.05,1.05)}, font=\small}]
			\addplot+[
			cycle  list name=trio,
			mark=square,
			]
			table [x=size,y=warp] {\Tthroughputduos};
			
			\addplot+[
			cycle  list name=trio,
			mark=square,
			]
			table [x=size,y=oha] {\Tthroughputduos};
			
			\legend{${s3p}_{dual}$, ${s3p}^ {*}$}
			
		\end{groupplot}
	\end{tikzpicture}
	\caption{Throughput of local s3p and remote s3p/MinIO instance}
\end{figure}

\subsection{Latency}

Measured with warp. 1 warm-up run. Results of the second run.

Measured with oha for a single object. Size 10KiB and 100KiB.


% local-minio
\pgfplotstableread{
	lw  lq  med  uq uw
	54	9	6	4	1
	44	10	7	4	1
	65	15	9	6	1
	60	31	25	20	4
	241	26	17	11	2
	
}\latlocalminio

% local-s3p
\pgfplotstableread{
	lw  lq  med  uq  uw
	10	 2	  1	  1	  0
	10	 2	  1	  1	  0
	11	 3	  2	  1	  0
	73	31	 22	  2	  0
	80	31	 22	 13	  0
}\latlocalsssp

% remote-minio
\pgfplotstableread{
	lw  lq  med  uq  uw
	25	5	4	2	1
	29	6	4	3	1
	42	8	6	4	1
	78	19	17	16	4
	37	16	15	14	7
}\latremoteminio

% remote-s3p
\pgfplotstableread{
	lw  lq  med  uq  uw
	9	2	1	1	0
	10	2	1	1	0
	11	3	2	1	0
	157	19	14	0	0
	62	46	39	25	0
}\latremotesssp

% remote-s3p-dual
\pgfplotstableread{
	lw  lq  med  uq  uw
	9	2	1	1	0
	10	2	1	1	0
	12	3	2	1	0
	66	18	7	1	0
	54	39	34	29	0
}\latremotedual

\begin{figure}[h!]
	\centering
	\begin{tikzpicture}
		\begin{axis}[
				xlabel=Time in ms,
				ylabel=Object size,
				enlarge x limits=0.05,
				minor x tick num=5,
				xminorticks=true,
				xmajorgrids=true,
				xbar interval=1,
				yticklabel style={
					text width=1.5cm,
					font=\small,
					align=center,
				},
				boxplot={
					draw position={1/3 + floor(\plotnumofactualtype/2) + 1/3*mod(\plotnumofactualtype,2)},
					box extend=0.3,
					cycle list name=duo,
				},
				ymin=0,
				ymax=5,
				xmin=0,
				%xmax=50,
				ytick={0,...,10},
				yticklabels={
					1KiB,
					10KiB,
					100KiB,
					500KiB,
					1MiB,
				},
				area legend,
				legend style={
					legend pos=south east,
					font=\small,
				},
				width=\textwidth,
				y=1cm,
			]
			\pgfplotstablegetcolsof{\latlocalminio}
			\pgfplotsinvokeforeach{0,...,\pgfplotsretval-1}{
				\addplot+[
					boxplot prepared from table={
						table=\latlocalsssp,
						row=#1,
						lower whisker=lw,
						upper whisker=uw,
						lower quartile=lq,
						upper quartile=uq,
						median=med
					}, boxplot prepared
				]
				coordinates {};
				
				\addplot+[
					boxplot prepared from table={
						table=\latlocalminio,
						row=#1,
						lower whisker=lw,
						upper whisker=uw,
						lower quartile=lq,
						upper quartile=uq,
						median=med
					}, boxplot prepared
				]
				coordinates {};
			}
			
			\legend{${s3p}_{local}$, ${MinIO}_{local}$}
	\end{axis}
	\end{tikzpicture}
	\caption{Latency of local s3p and local MinIO instance}
\end{figure}

\begin{figure}[h!]
	\centering
	\begin{tikzpicture}
		\begin{axis}[
			xlabel=Time in ms,
			ylabel=Object size,
			enlarge x limits=0.05,
			minor x tick num=5,
			xminorticks=true,
			xmajorgrids=true,
			xbar interval=1,
			yticklabel style={
				text width=1.5cm,
				font=\small,
				align=center,
			},
			boxplot={
				draw position={1/4 + floor(\plotnumofactualtype/3) + 1/4*mod(\plotnumofactualtype,3)},
				box extend=0.225,
				cycle list name=trio,
			},
			ymin=0,
			ymax=5,
			xmin=0,
			%xmax=50,
			ytick={0,...,10},
			yticklabels={
				1KiB,
				10KiB,
				100KiB,
				500KiB,
				1MiB,
			},
			area legend,
			legend style={legend pos=south east, font=\small},
			width=\textwidth,
			y=1.5cm,
			]
			\pgfplotstablegetcolsof{\latremoteminio}
			\pgfplotsinvokeforeach{0,...,\pgfplotsretval-1}{
				\addplot+[
				boxplot prepared from table={
					table=\latremotedual,
					row=#1,
					lower whisker=lw,
					upper whisker=uw,
					lower quartile=lq,
					upper quartile=uq,
					median=med
				}, boxplot prepared
				]
				coordinates {};
				
				\addplot+[
				boxplot prepared from table={
					table=\latremotesssp,
					row=#1,
					lower whisker=lw,
					upper whisker=uw,
					lower quartile=lq,
					upper quartile=uq,
					median=med
				}, boxplot prepared
				]
				coordinates {};
				
				\addplot+[
				boxplot prepared from table={
					table=\latremoteminio,
					row=#1,
					lower whisker=lw,
					upper whisker=uw,
					lower quartile=lq,
					upper quartile=uq,
					median=med
				}, boxplot prepared
				]
				coordinates {};
			}
			
			\legend{${s3p}_{dual}$,${s3p}_{remote}$,${MinIO}_{remote}$}
		\end{axis}
	\end{tikzpicture}
	\caption{Latency of local s3p and remote s3p/MinIO instance}
\end{figure}

\subsection{Bandwidth}

%local-minio
\pgfplotstableread{%
	colnames		0		1		2		3		4
	size			1 	   10 	  100	  500	 1000
	warp 		 2188	11485  103696  507011 1031319
	oha			 1794	10686	99165  488861 1005892
}\bandwidthlocmin
\pgfplotstabletranspose[colnames from=colnames]\Tbandwidthlocmin{\bandwidthlocmin}

%local-s3p
\pgfplotstableread{%
	colnames		0		1		2		3		4
	size			1 	   10 	  100	  500	 1000
	warp		   15	   26	  282  346968  875516
	oha 			0		0		4	   47	87775
}\bandwidthlocs
\pgfplotstabletranspose[colnames from=colnames]\Tbandwidthlocs{\bandwidthlocs}

\begin{figure}[h!]
	\centering
	\begin{tikzpicture}
		\begin{axis}[
			xlabel={Object size},
			ylabel={Avg. bytes transferred},
			enlarge y limits={{0.1,upper}},
			xmin=0,
			ymin=0,
			xtick={1,10,100,500,1000},
			ymode = log,
			log basis y=2,
			xmode = log,
			log basis x=10,
			legend style = {legend pos=outer north east, font=\small},
			ymajorgrids=true,
			grid style=dashed,
			scaled ticks=false,
			x tick label style = {font = \small, text width = 1.7cm, align = center, rotate = 70, anchor = north east},
			xticklabels={
				1KiB,
				10KiB,
				100KiB,
				500KiB,
				1MiB,
			},
			]
			
			\addplot+[
			cycle  list name=duo,
			mark=square,
			]
			table [x=size,y=warp] {\Tbandwidthlocs};
			
			\addplot+[
			cycle  list name=duo,
			mark=square,
			]
			table [x=size,y=warp] {\Tbandwidthlocmin};
			
			\addplot+[
			cycle  list name=duo,
			mark=square,
			]
			table [x=size,y=oha] {\Tbandwidthlocs};
			
			\begin{comment}
			\addplot+[
			cycle  list name=duo,
			mark=square,
			]
			table [x=size,y=oha] {\Tbandwidthlocmin};
			\end{comment}
			
			
			\legend{${s3p}_{local}$, 
				${MinIO}_{local}$, 
				${s3p}^{*}_{local}$, 
				%${MinIO}^{*}_{local}$
			}
			
		\end{axis}
	\end{tikzpicture}
	\caption{Bandwidth usage of local s3p and MinIO instance}
\end{figure}

\begin{figure}[h!]
	\centering
	\begin{tikzpicture}
		\begin{axis}[
			xlabel={Object size},
			ylabel={Avg. bytes transferred},
			enlarge y limits={{0.1,upper}},
			xmin=0,
			ymin=0,
			xtick={1,10,100,500,1000},
			ymode = log,
			log basis y=2,
			xmode = log,
			log basis x=10,
			legend style={
				legend pos = outer north east, 
				cells={align=left}
			},
			ymajorgrids=true,
			grid style=dashed,
			scaled ticks=false,
			x tick label style = {font = \small, text width = 1.7cm, align = center, rotate = 70, anchor = north east},
			xticklabels={
				1KiB,
				10KiB,
				100KiB,
				500KiB,
				1MiB,
			},
			]
			
			\addplot+[
			cycle list name=trio,
			mark=square,
			]
			coordinates {
				(1, 5) (10, 26) (100, 304) (500, 230152) (1000, 820637)
			};
			
			\addplot+[
			cycle list name=trio,
			mark=square,
			]
			coordinates {				
				(1, 5) (10, 26) (100, 284) (500, 324472) (1000, 911047)
			};
			
			\addplot+[
			cycle list name=trio,
			mark=square,
			]
			coordinates {
				(1, 2174) (10, 11514) (100, 104151) (500, 515719) (1000, 1036184)
			};
			
			\addplot+[
			cycle list name=trio,
			mark=square,
			]
			coordinates {
				(1, 0) (10, 0) (100, 5) (500, 40764) (1000, 96090)
			};
			
			\legend{${s3p}_{dual}$, ${s3p}_{remote}$, ${MinIO}_{remote}$, ${s3p}^{*}_{remote}$}
			
		\end{axis}
	\end{tikzpicture}
	\caption{Bandwidth usage of local s3p and remote s3p/MinIO instance}
\end{figure}
    \chapter{Summary}
    \chapter{Future Work}

To better understand the implications of these results, future studies could address the issues found with the testing method by devising a benchmark that evaluates the performance and utility of S3 servers and its caching behavoirs more accurately through a set of access patterns tha reflect those observed in the real-world more accurately. 
\\\\
Additionally, further research can be done on the benefits brought by the \textit{external composability} by exploring transmissions through protocols other than HTTP and further expanding on the idea of smart communications between two proxies.
    \chapter{Appendix}
    \end{thesis}

\end{document}