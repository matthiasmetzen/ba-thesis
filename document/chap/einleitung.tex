\chapter{Einleitung}

Nachfolgend einige Punkte, die es zu beachten gilt:
\begin{enumerate}
 \item Dieses Template dient als Vorlage, damit diverse Anforderungen wie allgemeines Layout der Arbeit,
       Formatierung etc. einheitlich sind. Weiterhin bietet es einen etwas leichteren Einstieg in LaTeX
       und sollte somit nicht zu viel Zeit in Anspruch nehmen, damit die eigentliche Arbeit nicht darunter leidet.
 \item Einzelne Kapitel dieser Vorlage geben einige grundlegende Konstrukte vor, die zum größten Teil per
       Copy/Paste übernommen werden können. Dennoch ist es unvermeidbar, sich mit LaTeX (bis zu einem
       gewissen Maß) zu beschäftigen. Mit Google kann man die meisten Probleme sehr leicht lösen und
       bekommt oft direkt den passenden Code geliefert.
\end{enumerate}

\clearpage

\section{Checkliste}

Folgende Punkte sollten vor der Abgabe der Arbeit unbedingt nocheinmal geprüft werden (keine Garantie auf Vollständigkeit!).

\begin{enumerate}
 \item[$\square$] Das Dokument ist auf doppelseitigen Druck ausgelegt, d.h. unbedingt doppelseitig drucken lassen.
 \item[$\square$] Vor dem Hochladen \textbf{alles} nochmal kontrollieren.
 \item[$\square$] Sind alle Kapitel vorhanden?
 \item[$\square$] Ist das Inhaltsverzeichnis vollständig?
 \item[$\square$] Sind alle Bestandteile der Arbeit vorhanden (Titelblatt, Inhaltsverzeichnis, Anhang, Erklärung, etc.)
 \item[$\square$] Sind leere Abschnitte entfernt (z.B. keine Algorithmen im Algorithmusverzeichnis)?
 \item[$\square$] Sind alle Todos aus dem Dokument entfernt und erledigt? 
 \item[$\square$] Steht das richtige Datum auf der Arbeit?
 \item[$\square$] Das Thema muss \textbf{exakt} so formuliert sein, wie bei der Anmeldung der Arbeit.
 \item[$\square$] Die Arbeit ist digital als PDF im Studierendenportal einzureichen.
 \item[$\square$] Jegliche digitalen Dokumente, Quellcode, Programme, Messergebnisse und Anleitungen für den Aufbau von Testumgebungen sowie der LaTeX-Quelltext zur Arbeit selbst müssen zum Zeitpunkt der Abgabe im GitLab Repository vorhanden sein.
\end{enumerate}