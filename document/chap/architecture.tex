\chapter{Architecture}
\label{architecture}

This section gives an overview of the general architecture and request handling of the software. Most of the internal design is described in terms of abstract \textit{components}, which form the building blocks of the software. There can be multiple implementations for components of any type that provide different functionality from each other.
To provide a high degree of flexibility and extensibility, all components are only very loosely coupled, with very limited interaction points between individual components. This is achieved by defining a unified internal representation for requests and responses. All components involved in request handling have their API defined in terms of these two types, thus allowing components to easily be combined or new components to be added without breaking compatibility. For easier handling, both types can also be extended with additional data to be passed between components.
\\\\
Requests are generated by a \code{Server} component, which serves as an entrypoint to the proxy. The protocol and format through which external requests can be made to a \code{Server} are implementation specific to a given \code{Server}. When an external request is made to a \code{Server}, the server creates the internal representation from it. It then calls a \textit{Handler} function that it has been passed. The handler function returns a future that asynchronously resolves the request. Upon completing the future, the server then transforms the resolved response into a response accepted by its underlying protocol and resolves the incoming request.
\\\\
Through the \code{Handler} function, the request is passed to a number of \code{Middleware} components. These \code{Middleware}s are responsible for operating on the request and providing additional capabilities.
Middlewares have access to both the request and response of every request and can intercept, replace or modify them.
The cache, one of the focuses of this work, has been implemented as one such middleware. Upon receiving the request, it checks for the cachability and attempts to retrieve an appropriate response from the cache. The cachability and validity of a cached response are validated using standardized HTTP cache policies. In case a matching response was found, the middleware short-circuits and no further middleware will be executed. 
\\\\
If the request has successfully passed through the middleware stack, it is passed to a \code{Client} component, which is responsible for resolving the request into a response. Like the \code{Server} component, the method used to resolve the request is dependent on the specific implementation and does not need to match the \code{Server} component that was used to accept the request.
\\\\
To enable additional interaction with the proxy, in addition to the described request-response pattern a \code{Webhook} component can be configured to receive external events. These events are broadcast to all active components, allowing them to further enhance their capabilities.
\\\\

\begin{figure}
	\label{diagram_architecture}
	\centering
	\begin{tikzpicture}
		\matrix (m) [matrix of nodes,
		column sep=5mm,
		row sep=1cm,
		nodes={draw, % General options for all nodes
			line width=1pt,
			anchor=center, 
			text centered,
			rounded corners,
			minimum width=1.5cm, minimum height=8mm
		}, 
		% Define styles for some special nodes
		broadcast/.style={cylinder, sharp corners, scale=0.5},
		other/.style={ellipse, scale=0.5},
		server/.style={isosceles triangle, anchor=center},
		middleware/.style={rectangle, rotate=90, scale=0.8},
		client/.style={dart},
		webhook/.style={trapezium, shape border rotate=90},
		empty/.style={draw=none, minimum width=0mm}
		]
		{
			|[empty]| {}
			& |[empty]| {}
			& |[empty]| {}
			& |[empty]| {}
			& |[empty]| {}
			& |[broadcast]| {Broadcast}
			& |[webhook]| {Webhook} 
			& |[empty]| {}
			\\
			|[empty]| {}
			& |[server]| {Server}
			& |[middleware]| {Middleware}
			& |[empty]| {...}
			& |[middleware]| {Middleware}
			& |[client]| {Client}
			& |[empty]| {}
			& |[empty]| {}
			\\
		};
		
		{
			\draw[<->, dashed, line width=1pt] (m-2-1.west) -- (m-2-2.west);
			\path[->, line width=1pt] (m-2-2) edge (m-2-3);
			\path[line width=1pt] (m-2-3) edge (m-2-4);
			\path[->, line width=1pt] (m-2-4) edge (m-2-5);
			\path[->, line width=1pt] (m-2-5) edge (m-2-6);
			\draw[<->, dashed, line width=1pt] (m-2-6.east) -- (m-2-8.east);
			
			\draw[<-, line width=1pt, rounded corners] (m-1-7.west) -- (m-1-6.east);
			\draw[<->, line width=1pt, rounded corners] (m-1-6.west) -| (m-2-2.north);
			\draw[<->, line width=1pt, rounded corners] (m-1-6.west) -| (m-2-3.east);
			\draw[<->, line width=1pt, rounded corners] (m-1-6.west) -| (m-2-5.east);
			\draw[<->, line width=1pt, rounded corners] (m-1-6.south) -- (m-2-6.north);
			
			\draw[<-, dotted, line width=1pt, rounded corners] (m-1-7.east) -- (m-1-8.east);
			\draw[->, dotted, line width=1pt, rounded corners] (m-1-6.west) -- (m-1-1.west);
			
			\tikzset{blue dotted/.style={draw=blue!50!white, line width=1pt,
					dash pattern=on 1pt off 4pt on 6pt off 4pt,
					inner sep=5mm, rectangle, rounded corners}};
				
			\node (pipeline box) [blue dotted, fit = (m-2-2) (m-1-7)] {};
			\node at (pipeline box.north) [below, inner sep=3mm] {\textbf{Pipeline}};
			
			\node (stack box) [blue dotted, fit = (m-2-3) (m-2-5)] {};
			\node at (stack box.north) [below, inner sep=3mm] {\textbf{Stack}};
		
		};
		
		\node (downstream) [rectangle, draw, rotate=90, left=of m, minimum width=6cm, anchor=center] {Downstream};
		\node (downstream) [rectangle, draw, rotate=90, right=of m, minimum width=6cm, anchor=center] {Upstream};
	\end{tikzpicture}
	\caption{Diagram of the architecture}
\end{figure}