\chapter{Fundamentals}

\section{RESTful API}

\section{Simple Storage Service}
\label{S3_desc}
Simple Storage Service (S3) is a cloud storage service originally developed by Amazon Web Services (AWS) that offers reliable and scalable cloud storage for data through a HTTP based API \cite{AWS_S3_Release} \cite{S3_API}. Since its initial release in 2006, many alternative implementations and competing S3-compatible services have been developed. Although the term S3 originally only referred to the service offered by AWS, it has now often used as a general term for any compatible solution.
In S3, individual data entries are called \textit{objects}, which consist of the \textit{value}, which can consist of any sequence of bytes ranging from 0 to 5 TB in size\cite{AWS_S3_Object}, and \textit{metadata}, a set of key-value pairs that stores additional information about the object. Objects are grouped in a container structure called \textit{buckets}, where each object belongs to exactly one bucket, and every bucket can contain any number of objects. Within a bucket, each object is uniquely identified through a \textit{key} and a \textit{version ID}.
The data stored in S3 can be accessed and managed through a HTTP-based REST API, which currently defines 97\cite{S3_API_Actions} operations through which data can be accessed and manipulated.
Access to the REST API is configured through policies that can be defined for each bucket which enable granular control over which operation can be executed by which user. The way user accounts are managed differs between implementations, but generally it can be said that users will be associated with \textit{credentials} consisting of an \textit{access key id} and a \textit{secret key}. To authorize requests to the API, users need to provide a signature for each request. This signature is calculated using multiple points of data like the contents of the request, a timestamp at which the request was made, as well as the user's credentials. The signature can be sent either through the request's \textbf{Authorization} header or its URL and is validated by the server.

With the ever-increasing amount of data that companies manage and the inherent need for storage capacity, S3 offers an easily manageable abstraction where developers do not have to think about the underlying filesystems and storage mediums. The decoupling of storage and access also makes it possible to expand storage capacity by spreading objects over multiple devices and even adding more capacity when necessary, allowing for near-infinite scalability. With managed solutions like the one provided by AWS, pricing is commonly calculated only by the amount of storage in use, which allows users to build solutions that can effortlessly scale to large amounts of data without wasting money on unused capacity caused by overprovisioning.

\section{MinIO}
MinIO is an implementation of a S3-compatible object store. It supports most of the S3 core features. Its greatest advantage is that it offers various means to create self-hosted instances, making it not only versatile but also favored by companies in regard to data protection.

\section{Composable Proxy}
\label{composition}
In this work, the term \textit{composable proxy} is used to describe proxy servers that are both \textit{internally composable} by dynamically combining internal components through simple configuration, as well as \textit{externally composable} by allowing multiple proxy servers to work together in order to enhance their abilities.

\section{Caching}

Caching is an important technique in computing to enhance the performance of computer systems. By storing frequently accessed data in a fashion that is faster to access, both the latency and throughput of applications can be greatly improved.
This section gives a brief overview of some of the terms used to describe caching strategies and outlines the cache policies used within this work.

\subsection{Admission Policy}
The admission policy defines the logic employed by a caching system to determine whether a resource should be stored in the cache or not. It can be influenced by various factors, such as the cache capacity, the size of a resource or its relevance.

\subsection{Eviction Policy}
The eviction policy describes the mechanism by which a caching system selects items to be removed from the cache when new items are inserted into a cache that has already reached its maximum capacity. Popular eviction policies include the Least Recently Used (LRU), and Least Frequently Used (LFU) strategies.

\subsection{TinyLFU}
\label{tiny_lfu}
TinyLFU is \textquote[{\cite{einziger2015tinylfu}}]{an approximate frequency based cache admission policy \textelp{that} can augment caches of arbitrary eviction policy and significantly improve their performance. \textelp{T}he memory consumption of TinyLFU \textins{was optimized} using adaptation of known approximate counting techniques with novel techniques tailored specifically in order to achieve low memory footprint for caches.}
In TinyLFU \textquote[{\cite{einziger2015tinylfu}}]{an accessed item is only inserted into the cache if an admission policy decides that the cache hit ratio is likely to benefit from replacing it with the cache victim \textelp{} chosen by the cache’s replacement policy}.

\section{HTTP Caching}
\label{http_cache_policy}

HTTP (Hypertext Transfer Protocol) caching is an important mechanism that allows web clients and intermediaries to store and reuse previously requested responses. Caching plays a vital role in improving the responsiveness of web applications and reducing bandwidth usage. To control the caching behavior of web resources, a set of HTTP headers was first defined as part of the HTTP/1.1 protocol\cite{rfc2616}. These headers provide instructions to clients and intermediate caching servers on how to handle and store cached content.

This section provides an overview of the most commonly used HTTP headers related to caching and their role in cache applications.

\subsection{Cache-Control} The Cache-Control header is the most fundamental directive that defines caching behavior. It specifies whether a resource can be cached, for how long it can be cached and whether a resource is allowed to be stored in a shared cache or only in a private cache. This header provides various options such as "no-cache" (attempt to validate the resource on every request), "no-store" (a cache must not store any part of the request or the response), "max-age" (set the maximum amount of time a resource may be cached), and "public" (allow caching in shared caches)\cite{rfc9111}.

\subsection{Expires} The Expires header indicates a timestamp after which a response should be considered stale. If a Cache-Control header with the "max-age" directive is present, the Expires header will be ignored\cite{rfc9111}.

\subsection{ETag} The ETag header provides a unique identifier that a server can assign to a specific version of a resource. It enables the server or cache to determine if a cached resource is still valid or if it conditionally needs to be refreshed. By augmenting subsequent requests to a resource with the If-None-Match header containing the previous ETag, the server can avoid unnecessary data transfer by responding with a "304 Not Modified" status\cite{rfc9111}.

\subsection{Last-Modified} The Last-Modified header provides a timestamp of when a resource was last modified on the server. By augmenting subsequent requests with the Last-Modified-Since header containing the previous Last-Modified value, the server can determine if a cached resource is still valid and conditionally respond with a "304 Not Modified" status\cite{rfc9111}.

\subsection{Vary} Servers use the Vary header to let intermediaries and caches know that the response might change depending on the values of specific request headers. It defines a set of headers that must match between subsequent requests and the original request used to retrieve a specific resource. If any of the headers do not match, the cached response may not be used to serve the request\cite{rfc9111}.

\section{Rust concepts}

The following points explain some basic concepts for the Rust programming language. While not necessarily unique to Rust, some of its concepts are not common to many other programming languages.
\begin{enumerate}
	\item Trait
	\item Enum Dispatch
\end{enumerate}

\section{Similar Software}
\subsection{SeaweedFS}
\subsection{Alluxio}
\subsection{Noobaa}