\chapter{Tabellen}

Tabellen sind vor allem im Abschnitt der Evaluation sehr wichtig. Die Messergebnisse aus der Tabelle müssen im Text erklärt und referenziert werden.

\section{Einfache Tabelle}
\begin{table}[H]
\begin{center}
    \begin{tabular}{| l | l | l | l | l |}
    \hline
    Anzahl der Iterationen & 100 & 250 & 500 & 1.000 \\ \hline
    Implementierung 1 (ms) & 0,375738 & 0,51265 & 0,77477 & 1,201552 \\ \hline
    Implementierung 2 (ms) & 0,397062 & 0,397987 & 0,411314 & 0,415611 \\ \hline
    \end{tabular}
\end{center}
\caption{Durchschnittliche Zeit pro Frame aus 10.000 Frames in Millisekunden}
\label{tabelle_avarage_time}
\end{table}

Die Ergebnisse aus Tabelle \ref{tabelle_avarage_time} zeigen, dass\ldots\\

\section{Komplexere Tabelle: Multicolumn und Multirow}

\begin{table}[H]
\begin{center}
    \begin{tabular}{| l | l | l | l | l |}
    \hline
    \multirow{2}{*}{$Datensatz 1$} & \multicolumn{2}{|c|}{$Datensatz 2$}  &
    \multicolumn{2}{|c|}{$Typ$} \\
    \cline{2-5}
    & $t_{total}(ms)$ & \#Pakete & $t_{total}(ms)$ & \#Pakete \\    		
    \hline 
    0 & 45,654 & 2 & 113,692 & 8 \\ \hline
    1 & 60,385 & 4 & 138,214 & 16 \\ \hline
    2 & 78,785 & 6 & 247,939 & 24 \\ \hline
    3 & 121,979 & 11 & 389,466 & 44 \\ \hline
    4 & 241,632 & 20 & 653,795 & 80 \\ \hline
    5 & 412,281 & 39 & 1.088,248 & 156 \\ \hline
    \end{tabular}
\end{center}
\caption{Gesamtzeiten der Übertragungen der Datensätze 1 und 2 in
Millisekunden der Typen 0 bis 5}
\label{tabelle_uebertragung}
\end{table}

Die Messergebnisse der Testreihe sind in Tabelle \ref{tabelle_uebertragung} festgehalten und zeigen\ldots


